%----------------------------------------------------------------------------------------
%	PACKAGES AND OTHER DOCUMENT CONFIGURATIONS
%----------------------------------------------------------------------------------------

\documentclass{resume} % Use the custom resume.cls style
\usepackage{multicol}
\usepackage{hyperref}
\setlength{\multicolsep}{6.0pt plus 2.0pt minus 1.5pt}
\setlength{\extrarowheight}{0.1cm}
\usepackage[left=0.75in,top=0.6in,right=0.75in,bottom=0.6in]{geometry} % Document margins

\name{Steven Shih} % Your name
% \address{6090 Iona Drive, 1007 \\ Vancouver, BC V6T 0B6} % Your address
\address{stevenshih1997@gmail.com} % Your secondary addess (optional)
\address{$+1(250)\hspace{0.05cm}\cdot\hspace{0.05cm}$885$\hspace{0.05cm}\cdot\hspace{0.05cm}2832$ \\ github.com/stevenshih1997} % Your phone number and email

\begin{document}

%----------------------------------------------------------------------------------------
%	WORK EXPERIENCE SECTION
%----------------------------------------------------------------------------------------
\begin{rSection}{Experience}

\begin{rSubsection}{Facebook}{June 2020 - Aug 2020}{Incoming Software Engineer Intern}{}
\end{rSubsection}

\begin{rSubsection}{Amazon}{Sep 2019 - Dec 2019}{Software Development Engineer Intern}{}
\item Designed and implemented a scalable batch endpoint solution to allow tenants to submit overrides to a program (Prime, Fresh, etc.).
\item Augmented feature with SQS to utilize a fleet of hosts to asynchronously process a large request.
\item Scaled the submission of overrides from a single override to the ability to submit millions.
\end{rSubsection}

\begin{rSubsection}{MetCredit}{Jun 2019 - Aug 2019}{Software Engineering Intern}{}
\item Processed imbalanced data using techniques such as over-sampling and SMOTE for training a predictive model.
\item Increased debt recovery rate using logistic regression analysis on client data to determine propensity to pay.
\end{rSubsection}

\begin{rSubsection}{Hootsuite}{Jan 2018 - Aug 2018}{Software Engineering Intern}{Inbox Team}
\item Utilized proto3 protocol buffers with Google's gRPC to extend a rootification service written in Golang.
\item Implemented and extended endpoints in GraphQL for interfacing with a React frontend.
\item Integrated Instagram Business capabilities with Hootsuite Streams.
\end{rSubsection}

\begin{rSubsection}{NTNU}{May 2016 - Aug 2016}{Research Intern}{}
\end{rSubsection}
\end{rSection}

%----------------------------------------------------------------------------------------
%	EDUCATION SECTION
%----------------------------------------------------------------------------------------

\begin{rSection}{Education}

{\bf UBC} \hfill {\em Expected 2021} \\ 
  Bachelor of Applied Science in Computer \& Electrical Engineering. \hfill {\em GPA: 3.7}

\end{rSection}

%----------------------------------------------------------------------------------------
%	PROJECTS SECTION
%----------------------------------------------------------------------------------------

\begin{rSection}{Projects}
\begin{multicols}{2}
  \begin{rSubsection}{\href{http://cpen-391.appspot.com}{Hotel Management API}}{}{Restify.js, Google Cloud SQL, Google Cloud filestore, Azure Cognitive Systems, Paypal API}{}
  \item REST API deployed on Google App Engine consuming several APIs for connecting to DE1-SOC and Raspberry Pi.
  \item Secure credit card handling with Paypal's Vault API.
  \item Facial recognition login using Azure Face API with file references to images saved in database and images saved in filestore.
\end{rSubsection}
\columnbreak %---------------------------------------------------------------------------
  \begin{rSubsection}{\href{https://github.com/stevenshih1997/Hotel-Serverless}{Smart Security Camera}}{}{Golang, Terraform, CloudFormation, AWS Lambda, AWS Kinesis, AWS Rekognition, AWS SNS}{}
  \item Terraform to efficiently initialize and deploy AWS infrastructure and code.
  \item AWS Kinesis and Rekognition to stream video and detect faces from Raspberry Pi camera.
  \item AWS Lambda and API Gateway provides HTTP endpoint when a trigger is activated.
  \item SNS to subscribe and notify users of intruders.
\end{rSubsection}
\end{multicols}

\end{rSection}
%----------------------------------------------------------------------------------------
%	TECHNICAL STRENGTHS SECTION
%----------------------------------------------------------------------------------------

\begin{rSection}{Skills}

\begin{tabular}{ @{} >{\bfseries}l @{\hspace{5ex}} l }
Programming Languages & Python, Java, JavaScript, Scala, Php  \\
Libraries/Frameworks & Express.js, Spring, React.js, Restify, ReactNative, Cats, Play  \\
Tools/Databases & Docker, Terraform, GraphQL, MySQL, PostgreSQL, Kubernetes \\
\end{tabular}

\end{rSection}


%----------------------------------------------------------------------------------------
%	ACTIVITIES SECTION
%----------------------------------------------------------------------------------------

\begin{rSection}{Activities}

\begin{tabular}{ @{} >{\bfseries}l @{\hspace{18ex}} l @{\hspace{6ex}} l}

NWHacks 2019 & Western Canada's largest hackathon & Top 20 (2019) \\
NWHacks 2017 & Western Canada's largest hackathon & Participant (2017) \\
UBC LaunchPad & Student-run software engineering team & Developer (2017 - Present) \\
\end{tabular}

\end{rSection}

%----------------------------------------------------------------------------------------
%	EXAMPLE SECTION
%----------------------------------------------------------------------------------------

%\begin{rSection}{Section Name}

%Section content\ldots

%\end{rSection}

%----------------------------------------------------------------------------------------

\end{document}
